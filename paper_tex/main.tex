\documentclass[conference]{IEEEtran}
\IEEEoverridecommandlockouts
% The preceding line is only needed to identify funding in the first footnote. If that is unneeded, please comment it out.
\usepackage{cite}
\usepackage{amsmath,amssymb,amsfonts}
\usepackage{algorithmic}
\usepackage{graphicx}
\usepackage{textcomp}
\usepackage{xcolor}
\def\BibTeX{{\rm B\kern-.05em{\sc i\kern-.025em b}\kern-.08em
    T\kern-.1667em\lower.7ex\hbox{E}\kern-.125emX}}
\begin{document}

\title{Verifying Sparse Matrix Computations
}

\author{\IEEEauthorblockN{1\textsuperscript{st} Given Name Surname}
\IEEEauthorblockA{\textit{dept. name of organization (of Aff.)} \\
\textit{name of organization (of Aff.)}\\
City, Country \\
email address or ORCID}
\and
\IEEEauthorblockN{2\textsuperscript{nd} Given Name Surname}
\IEEEauthorblockA{\textit{dept. name of organization (of Aff.)} \\
\textit{name of organization (of Aff.)}\\
City, Country \\
email address or ORCID}
\and
\IEEEauthorblockN{3\textsuperscript{rd} Given Name Surname}
\IEEEauthorblockA{\textit{dept. name of organization (of Aff.)} \\
\textit{name of organization (of Aff.)}\\
City, Country \\
email address or ORCID}
\and
\IEEEauthorblockN{4\textsuperscript{th} Given Name Surname}
\IEEEauthorblockA{\textit{dept. name of organization (of Aff.)} \\
\textit{name of organization (of Aff.)}\\
City, Country \\
email address or ORCID}
\and
\IEEEauthorblockN{5\textsuperscript{th} Given Name Surname}
\IEEEauthorblockA{\textit{dept. name of organization (of Aff.)} \\
\textit{name of organization (of Aff.)}\\
City, Country \\
email address or ORCID}
\and
\IEEEauthorblockN{6\textsuperscript{th} Given Name Surname}
\IEEEauthorblockA{\textit{dept. name of organization (of Aff.)} \\
\textit{name of organization (of Aff.)}\\
City, Country \\
email address or ORCID}
}

\maketitle

\begin{abstract}
Abstract of the paper goes here....
\end{abstract}

\begin{IEEEkeywords}
sparse matrix representations and computations, state-based verification, ...
\end{IEEEkeywords}

\section{Introduction}
In real-world applications, matrices are often sparse, meaning they mostly consist of zero-valued elements. There are many sparse matrix formats that compress large matrices into a more compact representation to reduce memory footprint and improve efficiency (see Appendix A). Implementation of sparse matrix computations thus gets obfuscated with complex loop structures with array indirections, in-place data mutation, and other low-level optimizations, making the code hard to read and even more difficult to verify. Furthermore, sparse matrix codes are affected by the issues of memory safety and full functional correctness.

Verification of sparse matrix computations is thus very crucial. However, this aspect has not been as well-studied as performance optimization strategies like data reordering and memory traffic reduction. Dyer et al. [1] present a state-based approach to perform bounded verification on the safety properties of sparse matrix computations using data abstraction and refinement principles. They use Alloy [3] to model the structure and behavior of sparse matrices and present an idiom to model nested loop structures with bounded iterations.

\subsection{Contributions}
\begin{itemize}
    \item We at first formed a good understanding of the various Alloy models [2] presented in the paper including the tabular idiom for specifying nested, bounded iterations. (Please refer to Appendix B for our documentation of existing code)
    \item We identified bugs in the existing Alloy models in the module for translation from ELL to the CSR format.
    \item We extended the Alloy models in that project to support:
\textbf{Translation from CSR to ELL}: We primarily wanted to explore the possibility of reusing the tabular idiom for other translations like CSR to ELL.
\textbf{Translation from COO to CSR}: This translation is straightforward, but it is crucial for many scientific applications and hence we modeled this translation.
\textbf{Sparse Matrix-Vector Multiplication with ELL}: We extended the existing CSR MVM model to support the ELL format.

\end{itemize}

\section{Background}

\section{Alloy Model Extensions}

\section{Discussion}

\section{Conclusions and Future Work}



\section*{Acknowledgment}

The preferred spelling of the word ``acknowledgment'' in America is without 
an ``e'' after the ``g''. Avoid the stilted expression ``one of us (R. B. 
G.) thanks $\ldots$''. Instead, try ``R. B. G. thanks$\ldots$''. Put sponsor 
acknowledgments in the unnumbered footnote on the first page.




\begin{thebibliography}{00}
\bibitem{b1} G. Eason, B. Noble, and I. N. Sneddon, ``On certain integrals of Lipschitz-Hankel type involving products of Bessel functions,'' Phil. Trans. Roy. Soc. London, vol. A247, pp. 529--551, April 1955.
\bibitem{b2} J. Clerk Maxwell, A Treatise on Electricity and Magnetism, 3rd ed., vol. 2. Oxford: Clarendon, 1892, pp.68--73.
\bibitem{b3} I. S. Jacobs and C. P. Bean, ``Fine particles, thin films and exchange anisotropy,'' in Magnetism, vol. III, G. T. Rado and H. Suhl, Eds. New York: Academic, 1963, pp. 271--350.
\bibitem{b4} K. Elissa, ``Title of paper if known,'' unpublished.
\bibitem{b5} R. Nicole, ``Title of paper with only first word capitalized,'' J. Name Stand. Abbrev., in press.
\bibitem{b6} Y. Yorozu, M. Hirano, K. Oka, and Y. Tagawa, ``Electron spectroscopy studies on magneto-optical media and plastic substrate interface,'' IEEE Transl. J. Magn. Japan, vol. 2, pp. 740--741, August 1987 [Digests 9th Annual Conf. Magnetics Japan, p. 301, 1982].
\bibitem{b7} M. Young, The Technical Writer's Handbook. Mill Valley, CA: University Science, 1989.
\end{thebibliography}
\vspace{12pt}


\end{document}
